\documentclass[12pt,a4paper]{article}
\usepackage[slovak]{babel}
\usepackage[utf8]{inputenc}
\usepackage[IL2]{fontenc}
\usepackage{verbatim}
\usepackage{a4wide}
\title{\bf Dokumentácia k programu \texttt{astsp}}
\date{\bf\today}
\author{\bf Peter Zeman}
\begin{document}
\maketitle

Program sa snaží riešiť Problém obchodného cestujúceho. To je problém, v ktorom máme na vstupe zadané mestá a cesty
medzi nimi, spolu s dĺžkami. Chceme nájsť cestu takú, ktorá začína a končí v jednom meste, obsahuje všetky mestá a
súčasne chceme, aby dĺžka takejto cesty bola minimálna.

Cieľom programu však nie je nájsť optimálnu cestu, ale v rozumnom čase sa k optimálnej ceste čo najviac priblížiť.
Program hľadá riešenie iteratívne, teda skončí po danom počte iterácií a počas každej iterácie sa snaží doposiaľ nájdené
riešenie vylepšiť.

Program sa ovláda jednoducho -- má dva povinné parametre. Prvý je počet iterácií, po ktorom má skončiť a druhý je názov
súboru, ktorý obsahuje popis miest. Volanie porgramu bude teda vyzerať takto:
\begin{verbatim}$ astsp <POCET ITERACII> <NAZOV SUBORU>\end{verbatim}
Mesto je v súbore reprezentované dvojicou celých čísel na samostatnom riadku. Program predpokladá, že existuje cesta
medzi kadždými dvoma mestami, a že jej dĺžka je ich euklidovská vzdialenosť, ktorú program spočíta zo zadaných
súradníc.

\end{document}
