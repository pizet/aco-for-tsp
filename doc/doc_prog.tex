\documentclass[12pt,a4paper]{article}
\usepackage[slovak]{babel}
\usepackage[utf8]{inputenc}
\usepackage[IL2]{fontenc}
\usepackage{a4wide}
\usepackage{graphicx}
\title{\bf Dokumentácia k programu \texttt{astsp}}
\date{\bf \today}
\author{\bf Peter Zeman}
\begin{document}
\maketitle

\section*{Problém obchodného cestujúceho}

Formálne môžeme TSP (traveling salesman problem) reprezentovať úplným
ohodnoteným grafom $G = (V, E)$, kde $V$ je množina vrcholov, ktoré reprezentujú
mestá a $E$ množina hrán, ktoré reprezentujú cesty medzi mestami. Každá cesta
$\{i,j\}\in E$ má priradenú dĺžku $d_{ij}$. Cieľom je nájsť hamiltonovskú
kružnicu s minimálnou dĺžkou. Optimálne riešenie TSP je teda permutácia $\pi$
vrcholov $\{1,\dots,n\}$ taká, že dĺžka $f(\pi)$ je minimálna, pričom $$f(\pi) =
\sum_{i=1}^{n-1}d_{\pi(i)\pi(i+1)} + d_{\pi(n)\pi(1)}.$$

Úlohou programu je priblížiť sa v rozumnom čase čo najviac k optimálnemu riešeniu TSP. Zavolá si na pomoc mravce!

\section*{Ant System}

Dve hlavné fázy algoritmu Ant System (AS) sú budovanie ciest pomocou mravcov a aktualizácia feromónov. Základná kostra
algoritmu vyzerá takto:

\begin{verbatim}
def aco_for_tsp():
    # Inicializacia
    initialize_data(...)

    while not terminate(...):
        construct_solutions(...) # Budovanie ciest
        best_tour_length, best_tour = update_statistics(...)
        update_pheromone_trails(...) # Aktualizacia feromonov
\end{verbatim}

Pri budovaní ciest je najprv každému mravcovi náhodne priradené nejaké štartovné mesto. Potom každý mravec z tohoto
mesta vybuduje cestu, ktorá spĺňa podmienky TSP (končí v štartovnom meste a súčasne prejde každým mestom práve raz).
Budovanie ciest prebehne v $n$ krokoch ($n$ je počet miest) a v každom kroku je pre každého mravca vybraté jedno mesto.
Po skončení tejto procedúry je zaznamenaná najlepšia nájdená cesta a sú aktualizované feromóny. Je viac možností, kedy
ukončiť celý algoritmus, napríklad keď sa dosiahne maximálneho počtu iterácií.

\subsubsection*{Výber mesta}

V každom kroku budovania ciest si musí mravec $k$ vybrať, ktoré bude jeho daľšie navštívené mesto. Mravec sa rozhodne na
základe pravdepodobnostného pravidla. Pravdepodobnosť, s ktorou si mravec $k$ v meste $i$ vyberie mesto $j$, je
$$p_{ij}^k=\frac{[\tau_{ij}]^{\alpha}[\eta_{ij}]^{\beta}}{\sum_{l\in N_i^k}[\tau_{il}]^{\alpha}[\eta_{il}]^{\beta}},
\ \ {\rm ak}\ j\in N_i^k,$$

kde $\eta_{ij} = 1/d_{ij}$ ($d_{ij}$ je vzdialenosť medzi mestami $i$ a $j$) je heuristická hodnota, $\tau_{ij}$ je
množstvo feromónov uložených medzi mestami $i$ a $j$. Konštanty $\alpha$ a $\beta$ sú parametre, ktoré určujú vplyv
feromónov a heuristickej informácie. Ak $\alpha = 0$, tak feromóny nemajú žiadny vplyv a dostávame obyčajný hladný
algoritmus. Ak je $\beta = 0$, tak majú vplyv len feromóny. Konečne $N_i^k$ značí množinu doposiaľ nenavštívených miest mravcom $k$ v meste $i$.
Podľa uvedeného vzorca, pravdebodobnosť výberu nejakej hrany $(i,j)$ rastie s hodnotami $\tau_{ij}$~a~$\eta_{ij}$.

Procedúra, ktorá vyberá nasledujúce mesto funguje nasledovne. Nech $c$ značí mesto, v ktorom sa práve nachádza mravec
$k$. Každá hodnota $choice\_info[c][j]$ ($choice\_info[i][j] =
[\tau_{ij}]^{\alpha}[\eta_{ij}]^{\beta}$), kde $j$ je nenavštívené mesto mravcom $k$, sa pripočíta k
$sum\_probabilities$. Položíme $selection\_probability[j] = choice\_info[c][j]$ ak mesto $j$
ešte navštívené, inak kladieme $selection\_probability[j] = 0$. Teraz zvolíme náhodne číslo $r$ z intervalu\\
$[0,sum\_probabilities]$. Potom začneme prechádzať polom $selection_probability$ a jednotlivé prvky nasčítavame do
premenenej $p$. V momente keď je $p\ge r$ prechádzanie zastavíme a mesto, pri ktorom sa tak stalo bude nasledujúce.

\begin{verbatim}
r = random.uniform(0.0, sum_probabilities)
p = 0.0
for i in range(len(selection_probability)):
    p += selection_probability[i] # Pripocitame selection probability
    if p >= r:
        # Vybrali sme mesto
        break
ants[ant].tour[step] = nn_list[city][i]
\end{verbatim}

\subsubsection*{Aktualizácia feromónov}

Posledným krokom je aktualizácia feromónov. Kostra tejto procedúry je takáto:

\begin{verbatim}
def update_pheromone_trails(ants, dist, pheromone, choice_info):
    # Najprv nechame feromony vyprchat
    evaporate(pheromone)
    # S kazdym mravcom aktualizujeme feromony na jeho ceste
    for i in range(len(ants)):
        deposit_pheromone(i, ants, pheromone)
    # Aktualizujeme choice_info
    create_choice_info_matrix(choice_info, dist, pheromone)
\end{verbatim}

Najprv necháme vyprchať feromóny, to znamená, že $\rho$-krát znížime hodnotu $\tau_{ij}$, kde $\rho$ je nejaká
konštanta. Potom za každého mravca pridáme feromóny na hrany, ktorými prešiel. Nech $tl$ značí dĺžku cesty, ktorú
prešiel nejaký mravec $k$. Potom mravec $k$ uloží na každú hranu, ktorou prešiel, $1/tl$ feromónov. Teda čim kratšiu
cestu mravec prešiel, tým viac feromónov uloží. Nakoniec stačí už len aktualizovať $choice\_info$.

\begin{thebibliography}{9}
        \bibitem{ant}{\em M. Dorigo, T. Stützle:}
                {\bf Ant Colony Optimization}\\
                The MIT Press,
                2004
\end{thebibliography}

\end{document}
